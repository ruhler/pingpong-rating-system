\documentclass{article}

\begin{document}
\title{A Ping Pong Rating System}
\author{Richard Uhler}
\maketitle

\section{Introduction}
This document describes a ping pong rating system for rating players in a
local league. The format of the league is that every month participating
players are grouped with approximately four other players to play against. The
ratings are used to choose the groups; players are grouped with other players
of a similar level.

We assume that the large majority of players are amateur, without any prior
ping pong rating and not competing nationally or otherwise outside of the
local league. In total we expect not more than a few hundred distinct players
to ever participate in the league. Individual players will play around four
matches a month, and there will not be more than around one hundred players
participating in any single month.

The primary goal of the rating system is to quickly establish a decent rank
order of players used to select the groups each month. Secondary goals of the
rating system are for the benefit and enjoyment of the players themselves:
\begin{itemize}
  \item A player should have a sense of how likely they are to win a match
    against another player by their ratings.
  \item For the most part, players should have unique ratings so that you can
    tell the difference between any two players.
  \item It is easy for players to remember and share their ratings with other
    players.
  \item Players are motivated to play more games rather than less games.
\end{itemize}

In a nutshell, we assume players have ratings such that the probability
of a player winning a match against another player is a logistics function of
the difference in their player ratings and that ratings are normally
distributed. We then estimate player ratings as those ratings that maximize
the likelihood of the observed match results. Finally, we scale the estimated
player ratings so that they are likely to be in the range from 0 to 2000, with
the average player having a rating of 1000.

\section{TODO}
* The Logistics Function - 1 / (1 + exp(-(x-y))) probability function
* Normal Distribution of Ratings - normal distribution of ratings (with simple motivation of one match)
* Maximum Likelihood function to maximize, transform it into minimize sum of squared error
* choice of sigma squared
* Scaling Factor scaling of ratings to $N(1000, 250^2)$
\end{document}
